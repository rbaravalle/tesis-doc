\chapter{Conclusiones y Trabajos a Futuro}
En este capítulo presentamos resumidamente las conclusiones de los distintos tópicos que fueron estudiados durante el desarrollo de las actividades de investigación del presente trabajo.
La tesis presenta cuatro áreas de investigación principales: el modelizado de geometrías tridimensionales de materiales porosos, su especialización a materiales específicos considerando procesos reales de formación, el renderizado de esas geometrías, y finalmente, el estudio de técnicas multifractales en la validación de los resultados.

En cuanto al modelado de geometrías porosas, el estudio comenzó con la introducción de sistemas de partículas en el modelado de texturas bidimensionales.
Dichas texturas resultaron ser una herramienta flexible en la representación de diversos materiales comunes, como madera o mármol.
Luego, se profundizó dicho modelado, extendiéndolo a tres dimensiones, a través del diseño de un sistema procedimental de generación de geometrías tridimensionales.
El sistema se basa en el crecimiento de burbujas que se evitan unas a otras, donde el crecimiento está regulado por un sistema dinámico.
De esta forma se obtiene una herramienta flexible y automática para representar una diversidad de sistemas porosos, como pan, esponjas y materiales similares.
Finalmente, se especializaron las técnicas al material pan, introduciendo un proceso completo de modelado procedimental, inspirado en el proceso de fabricación del pan, el cual obtiene imágenes foto-realistas del mismo, tanto en miga como corteza.
Esto es resultado de la utilización de parámetros físicos tanto en su formación ({\em e.g.}, parametrización de distribuciones de burbujas), como en su visualización ({\em e.g.}, distribución de colores en la corteza).
Es posible obtener imágenes en cualquiera de las etapas de formación (leudado, cocción, etc.), así como realizar cortes arbitrarios al material, resultando en un enfoque comprensivo al problema de modelado y renderización de pan, en computación gráfica.
Se demostró que la introducción de procesos de formación produjo un modelo superior a un enfoque puramente procedimental, ya que los parámetros obtenidos están en concordancia con materiales reales, permitiendo la eliminación de decisiones {\em ad-hoc}.

Se observó que un enfoque de tiempo real precisa, en las placas gráficas actuales, una textura volumétrica de dimensiones pequeñas (típicamente, $256^{3}$ voxels en cada eje).
Si se desea una imagen con mayor grado de realismo, se necesitan texturas de mayores dimensiones ($512^{3}$).
Sin embargo, esta limitación aparente fue parcialmente soslayada por medio de la utilización de más de una frecuencia en el muestreo de la textura, lo cual dio lugar a texturas con mayor detalle en tiempo real, utilizando texturas de baja resolución y agregando sólo un leve costo computacional.

El renderizado de estas geometrías fue abordado por medio de la utilización de técnicas basadas en la ecuación RTE (más precisamente, {\em Renderización de Volúmenes}), la cual modela la interacción de la luz con los objetos de una escena.
La utilización de dichas técnicas derivó en la obtención de imágenes foto-realistas, en tiempo real.
La GPU resulta indispensable en la obtención de dichas tasas de refresco ($> 30 $ FPS).
La implementación del algoritmo de renderizado tuvo lugar en la GPU a través de los denominados shaders (ver Apéndice A), de esta forma, es posible paralelizar el cómputo a nivel de píxel, obteniendo mejoras drásticas en el tiempo total de renderizado de una escena.
Si bien la técnica de renderizado no es novedosa, la aplicación de la misma al modelado de pan y geometrías porosas no había sido reportada.
Adicionalmente, en el caso del pan, la corteza del mismo es simulada utilizando la misma técnica, con diferentes parámetros (colores, reflectancia, translucencia, etc.), a diferencia del estado del arte, donde se los considera materiales diferentes.


Además, se realizó un estudio de las características fractales y multifractales del pan, como base para validar las estructuras e imágenes obtenidas.
En primer lugar se estudió el comportamiento mono-fractal de cortes de distintos tipo de panes, sin embargo, los resultados muestran la necesidad de utilizar más de una dimensión para caracterizar de manera adecuada los datos obtenidos.
Debido a este resultado, se aplicó una combinación de técnicas fractales, junto con el MFS, en la caracterización y clasificación de imágenes de miga de pan. Finalmente, se observó que la utilización, únicamente, del espectro multifractal (sin considerar características fractales) arroja los mejores resultados, superando clasificadores como SIFT \cite{Lowe2004}, junto a un costo computacional menor.
Adicionalmente, se observaron correlaciones entre determinadas dimensiones fractales del MFS y características típicas del pan, como la porosidad, la distribución de burbujas y el área media de burbuja, permitiendo concentrar toda la información en un único vector de características.
Finalmente, se validaron cortes bidimensionales de las geometrías obtenidas por el método inspirado en el proceso de fabricación del pan.
Por medio de una búsqueda exhaustiva, se encontraron valores en la generación cuyas características multifractales se corresponden, estadísticamente, con panes reales.
La búsqueda comienza por medio de la asignación manual de valores de los parámetros, intentando obtener imágenes visualmente similares a las reales.
Luego, un algoritmo automático compara exhaustivamente los MFS reales y sintéticos, devolviendo aquél conjunto de parámetros que minimizan el error, el cual se define como la diferencia entre las medianas de un conjunto de imágenes reales con las medianas de imágenes sintéticas.
Las imágenes resultantes muestran además una muy buena correspondencia visual entre las imágenes sintéticas y las reales, donde las sintéticas se generan utilizando los valores extraídos de las reales.

Las investigaciones en el modelizado, y el renderizado explicados a lo largo de este escrito, dieron lugar a trabajos de colaboración tanto con grupos de universidades nacionales, como internacionales.
El modelizado de las geometrías porosas fue abordado junto al grupo de Gráficos de la Universidad de Girona.
En dicho grupo se estudian técnicas de modelado procedimental de diversa índole, como por ejemplo ciudades y edificios.
El renderizado de los materiales derivó en un trabajo en conjunto con el Dr. Cristian García Bauza, del instituto PLADEMA, con Leonardo Scandolo, de la Universidad Nacional de Rosario y la Delft University of Technology, Holanda, y el Dr. Elmar Eisemann, perteneciente a la Delft University of Technology, Holanda.



\section{Trabajos a Futuro}
En estos momentos nos encontramos definiendo varias líneas de investigación las cuales surgen naturalmente de los resultados de la presente Tesis.
Por un lado, el modelado procedimental de otros objetos porosos, como huesos, está siendo estudiado.
Los huesos presentan en su interior una estructura esponjosa, la cual puede recibir el mismo tratamiento que obtuvo la miga de pan, tanto en modelado como renderización.
Además, la capa externa del hueso puede seguir el mismo camino que la corteza del pan, logrando un tratamiento integrado de todas las capas del hueso.
Siguiendo esta línea, el modelado y renderizado de cerámicos también puede beneficiarse de estas experiencias pioneras, por lo cual ya se han realizado reuniones con el grupo de cerámicos del instituto de física de Rosario (IFIR), liderado por la Dra. Nora Pellegri, con vistas a realizar intercambios de imágenes microscópicas de dichos materiales.

Por otro lado, el renderizado está siendo extendido, por medio de la toma en consideración del término de difusión interna ({\em subsurface scattering} en inglés), oclusión de ambiente basada en consideraciones físicas, y una mejor integración de las distintas escalas presentes en los materiales porosos.
Además, se está llevando a cabo la implementación de una transición suave entre los materiales de corteza y miga, aumentando el grado de realismo tanto en el pan como en los materiales que utilicen más de una capa.

El estudio fractal y multifractal, y la validación de las imágenes también presenta diversas continuaciones posibles.
Por un lado, en los últimos años surgieron algoritmos de aprendizaje profundo ({\em deep learning} en inglés \cite{Kiros2014}), los cuales cuentan con facilidades en la web para clasificar imágenes.
Determinadas imágenes renderizadas en el presente trabajo ya obtuvieron la clasificación de ``French loaf" (pan baguette) en un sistema online basado en dicha metodología\footnote{Disponible en http://deeplearning.cs.toronto.edu/}, el cual clasifica automáticamente imágenes entre miles de clases.
Esto demuestra que la validación visual, objetiva, de los materiales que se pretendían alcanzar están en concordancia con los reales.
Se busca profundizar esta línea de investigación, y extenderla a otros materiales.
Además, se busca automatizar la obtención de parámetros en el algoritmo de generación de texturas volumétricas.
De esta manera, será posible la obtención de parámetros de generación en materiales como esponjas, huesos, cerámicos y similares.
También buscamos obtener una mayor comprensión de las dimensiones multifractales, buscando entender su relación con características observables de los materiales.
De esta manera sería posible obtener una generación procedimental a partir de un MFS dado.
Por otro lado, el MFS, debido a su bajo costo computacional, puede ser utilizado en cadenas de producción, donde sería posible asociar parámetros de calidad de materiales con determinados MFS, de esta manera detectando materiales defectuosos en tiempo real.
La implementación en GPU de los algoritmos de generación que aún se realizan en CPU, podrían dar lugar a una mejora en los tiempos de cómputo de las geometrías.
Además, podría explorarse la generación directa de las geometrías en la placa gráfica a medida que se renderiza el material.

