\chapter{Conclusiones y Trabajos a Futuro}
En el presente capítulo presentamos resumidamente las conclusiones de los distintos tópicos que fueron estudiaros durante el desarrollo de las actividades de investigación del presente trabajo.
La tesis presenta tres áreas de investigación principales: por un lado, el modelizado de geometrías tridimensionales de materiales porosos, por otro, el renderizado de esas geometrías, y finalmente, el estudio de técnicas multifractales en la validación de los resultados.

En cuanto al modelado de geometrías porosas, el estudio comenzó con la introducción de sistemas de partículas en el modelizado de texturas bidimensionales.
Dichas texturas resultaron en una herramienta flexible en la representación de diversos materiales comunes, como madera o mármol.
Luego, se profundizó dicho modelado, extendiéndolo a tres dimensiones, a partir del diseño de un sistema procedimental de generación de geometrías tridimensionales.
El sistema se basa en el crecimiento de burbujas que se evitan unas a otras, donde el crecimiento está regulado por un sistema dinámico.
De esta forma se obtiene una herramienta flexible y automática para representar una diversidad de sistemas porosos, como pan, esponjas y otros materiales similares.
Finalmente, se introdujo un proceso completo de modelado procedimental inspirado en el proceso de fabricación del pan, el cual obtiene imágenes foto-realísticas del mismo, tanto en miga como corteza, gracias a la utilización de parámetros físicos tanto en su formación, como en su visualización (distribución de colores en la corteza, parametrización de distribuciones de burbujas, etc).
Es posible obtener imágenes en cualquiera de las etapas de formación (leudado, cocción, etc.), así como realizar cortes arbitrarios al material, resultando en un enfoque comprensivo al problema de modelado y renderización de pan, en computación gráfica.

El renderizado de estas geometrías, ...

La validacion de los resultados, ...

La investigación en el modelizado, y el renderizado explicados a lo largo de este escrito, dieron lugar a trabajos de colaboración tanto con grupos de universidades nacionales, como internacionales.
El modelizado de las geometrías porosas fue abordado junto al grupo de Gráficos de la Universidad de Girona.
En dicho grupo se estudian técnicas de modelado procedimental de diversa índole, como por ejemplo ciudades y edificios.
El renderizado de los materiales derivó en un trabajo en conjunto con el Dr. Cristian García Bauza, del instituto PLADEMA, y con Leonardo Scandolo, de la Universidad Nacional de Rosario.



\section{Trabajos a Futuro}

