\chapter{Introducción}
El renderizado en computación gráfica es el área que estudia métodos para producir imágenes que representan una escena tridimensional, la cual está compuesta por medio de primitivas matemáticas como puntos, líneas, cubos, etc.

En los últimos años, el avance en el campo del renderizado de escenas ha sido muy notorio. 
El nivel de realismo presente en las imágenes obtenidas ha ido en aumento hasta el punto de ser de difícil distinción, para un ser humano, si una imagen renderizada es real o sintética.
Dichos avances han sido debidos en gran medida al desarrollo de dispositivos de hardware gráficos más poderosos, ya que la aproximación matemática a la teoría física que describe el comportamiento de la luz en escenas fue desarrollada hace varias décadas, mediante la denominada Ecuación del Renderizado \cite{Kajiya1986}.
Sin embargo, los tiempos computacionales requeridos en la obtención de una imagen realista típicamente se extendían durante horas, resultando en una teoría completa pero poco práctica. 

La ecuación constituye una aproximación al fenómeno físico de interacción de la luz con diversos objetos en una escena.
Lamentablemente, la ecuación resulta, en términos teóricos, no computable.
Sin embargo, a lo largo de los años, se desarrollaron diversas técnicas que permiten aproximar diferentes aspectos de la misma (reflectancia, términos difusos, etc.).
Esto ha resultado determinante en la obtención de imágenes con un realismo asombroso, y tiempos computacionales aceptables.

A pesar de estos increíbles avances en un corto período de tiempo, el renderizado de una escena es un problema que involucra otras consideraciones.
Entre ellas, la ecuación del renderizado no tiene en cuenta {\em cuálesd materiales} componen los objetos que definen la escena.
Un objeto de una escena cuyo material se define como madera, no lucirá exactamente igual que uno de metal, ni a un objeto cuyo material está compuesto por tejidos.
Debido a esta necesidad, a la par del desarrollo de técnicas de iluminación global, surgieron técnicas que intentaron abordar materiales específicos, o familias de materiales.

Estas técnicas buscan capturar la intrincada geometría propia de cada material.
Diferentes estructuras microscópicas producen distintas apariencias, reflejando la luz de distinta manera, hecho que es interpretado por la percepción humana como objetos con diferentes materiales.
Por ejemplo: una superficie metálica tiene un gran componente reflexivo, emitiendo gran parte de su energía en direcciones específicas, a diferencia de una superficie más opaca como un plástico, la cual no presenta direcciones principales de emisión de radiancia.
Dado que la ecuación del renderizado utiliza a los materiales como {\em caja negra}, los mismos deben ser modelados de una manera adecuada, específica para cada material, para luego ser integrados en las distintas técnicas de renderizado global.

Determinados materiales han recibido mayor atención debido a su ubicuidad en escenas de películas y video juegos, o a la facilidad de su diseño: agua, fuego, aire, humo, piel humana, madera, etc.
Por esta razón, las imágenes sintéticas obtenidas presentan cierta asimetría en la calidad de los distintos materiales.
En contraste, otros materiales han recibido menor atención, la cual puede atribuirse a una menor presencia en escenas o a una dificultad en el modelado y visualización, la cual ha resistido las técnicas más simples.

Entre estos materiales, los materiales porosos o heterogéneos, y aquellos sometidos a un proceso de cocción han permanecido entre los más dificultosos por su compleja geometría y los fenómenos lumínicos involucrados.
Tal vez el caso más emblemático de los materiales cocidos es el pan, debido a su importancia en la vida cotidiana.
Como nota de color, un reconocido  científico del área, Alain Fournier, pronunció en 2001 la frase "la Computación Gráfica todavía no ha sido capaz de renderizar de manera convincente una feta de pan".
En las siguientes secciones mostraremos que unos pocos intentos por abordar el problema fueron llevados a cabo pocos años después de la emisión de dicho comentario.

\section{Alcances y Objetivos}
Esta tesis propone el estudio e implementación de algoritmos tanto en CPU como GPU, que permitan un modelado y renderización adecuados y realistas de materiales porosos y cocidos.
Como material principal, la presente tesis tomará al {\em pan}, ya que el mismo se encuentra en la intersección de ambas familias, permitiendo derivar algoritmos en ambos sentidos.
Particularmente, se proponen algoritmos procedimentales (es decir, sin intervención del usuario) de generación de la geometría interna y externa (miga y corteza) del pan.
Además, se propone una representación adecuada, junto a un algoritmo de renderizado en GPU para las geometrías resultantes.
Finalmente, se validan los resultados utilizando técnicas multifractales.
Luego estos algoritmos se derivan a otros materiales.

\subsection{Aplicaciones}
Los algoritmos desarrollados poseen diversas aplicaciones en el campo del modelado procedimental, y en el renderizado de materiales.
Más específicamente, pueden mencionarse:

\begin{itemize}
\item {\bf Modelado de materiales con geometrías porosas microscópicas y macroscópicas}: las técnicas fractales desarrolladas capturan eficazmente variadas geometrías porosas. Es posible utilizar el algoritmo semi-automático de búsqueda de parámetros presentado, en la representación de otras geometrías macro-porosas. Estos resultados podrían generalizarse, y aplicarse en el desarrollo de técnicas de Monte-Carlo para la obtención de distribuciones estadísticas de radiancia de materiales con superficie microscópicamente porosa.
\item {\bf Visualización médica realista}: la utilización de volúmenes en la representación de los materiales permite su aplicación en contextos médicos, donde se cuenta con datos provenientes de tomografías computadas, las cuales presentan materiales porosos como huesos.
\item {\bf Extracción de características de materiales}: los métodos multifractales utilizados pueden ser aplicados en la discriminación de muestras reales de otros materiales, así como en la caracterización de los mismos. Estos estudios pueden devolver parámetros útiles en la generación procedimental de los mismos.
\item  {\bf Clasificación de imágenes de materiales}: el espectro multifractal demostró ser útil en la clasificación y caracterización de imágenes de miga de pan. Es posible que el mismo sea útil en la clasificación de imágenes de materiales similares, pudiendo utilizarse al método en cadenas de producción, o en la determinación de parámetros de calidad del material.
\end{itemize}

\section{Resultados Originales Presentados}
Para lograr estos objetivos, se tomó en consideración el proceso de cocción del pan, y de manera más general, se estudió la cadena completa de pasos existente en la elaboración del mismo (leudado, cocción, etc.).
Debido a que la literatura de ingeniería de los alimentos presenta estudios muchas veces inconexos, sobre los distintos pasos involucrados, se implementaron algoritmos que aproximan los pasos más preponderantes: leudado y cocción, dejando de lado otros procesos como el amasamiento del mismo, o la formación de la corteza.

Resumidamente, las contribuciones más preponderantes incluyen:
\begin{itemize}
\item un algoritmo fractal que induce una distribución de radios de burbujas consistente con el proceso de leudado de la masa cruda del pan,
\item un algoritmo que aproxima el proceso de cocción del pan,
\item un algoritmo que aproxima el efecto de levantamiento del pan durante la cocción, basado en las distribuciones de burbujas existentes en el leudado,
\item una técnica que permite asignar colores en la corteza, utilizando el espacio de color CIELab, indicando mayor o menor cocción en cada punto,
\item el estudio del espectro multifractal en la discriminación de imágenes de migas de pan, y,
\item un algoritmo que utiliza el espectro multifractal en la aproximación de una distribucioń de burbujas dada.
\end{itemize}

