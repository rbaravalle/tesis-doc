\chapter{Introducción}
El renderizado en computación gráfica es el área que estudia métodos para producir imágenes que representan una escena tridimensional, la cual está compuesta por medio de primitivas matemáticas como puntos, líneas, cubos, etc.
En los últimos años, el avance en el campo del renderizado de escenas ha sido muy notorio. 
El nivel de realismo presente en las imágenes obtenidas ha ido en aumento hasta el punto de ser de difícil distinción, para un ser humano, si una imagen renderizada es real o sintética.
Dichos avances han sido debidos en gran medida al desarrollo de dispositivos de hardware gráficos más poderosos, ya que la aproximación matemática a la teoría física que describe el comportamiento de la luz en escenas fue desarrollada hace varias décadas, mediante la denominada Ecuación del Renderizado \cite{Kajiya1986}.
Sin embargo, los tiempos computacionales requeridos en la obtención de una imagen realista típicamente se extendían durante horas, resultando en una teoría descriptivamente completa, pero computacionalmente poco práctica. 

La ecuación constituye una aproximación al fenómeno físico de interacción de la luz con diversos objetos en una escena.
Lamentablemente, la ecuación resulta, en términos teóricos, no computable.
Sin embargo, a lo largo de los años, se desarrollaron diversas técnicas que permiten aproximar diferentes aspectos de la misma (reflectancia, términos difusos, etc.).
Esto ha resultado determinante en la obtención de imágenes con un realismo asombroso, y tiempos computacionales aceptables.

A pesar de estos increíbles avances en un corto período de tiempo, el renderizado de una escena es un problema que involucra otras consideraciones.
Entre ellas, la ecuación del renderizado no tiene en cuenta {\em cuáles materiales} componen los objetos que definen la escena.
Un objeto de una escena cuyo material se define como madera, no lucirá exactamente igual que uno de metal, ni a un objeto cuyo material está compuesto por tejidos.
Debido a esta necesidad, a la par del desarrollo de técnicas de iluminación global, surgieron técnicas que intentaron abordar materiales específicos, o familias de materiales.

Estas técnicas buscan capturar la intrincada geometría propia de cada material.
Diferentes estructuras microscópicas producen distintas apariencias, reflejando la luz de distinta manera, hecho que es interpretado por la percepción humana como objetos con diferentes materiales.
Por ejemplo: una superficie metálica tiene un gran componente reflexivo, emitiendo gran parte de su energía en direcciones específicas, a diferencia de una superficie más opaca como un plástico, la cual no presenta direcciones principales de emisión de radiancia.
Dado que la ecuación del renderizado utiliza a los materiales como {\em caja negra}, los mismos deben ser modelados de una manera adecuada, específica para cada material, para luego ser integrados en las distintas técnicas de renderizado global.

Determinados materiales han recibido mayor atención debido a su ubicuidad en escenas de películas y video juegos, o a la facilidad de su diseño: agua \cite{Schechter2012}, fuego, aire, humo, piel humana \cite{Donner2006}, madera, plásticos \cite{Kurt2010}, etc.
En contraste, otros materiales han recibido menor atención, la cual puede atribuirse a una menor presencia en escenas o a una dificultad en el modelado y visualización, la cual ha resistido las técnicas más simples.
Por esta razón, las imágenes sintéticas obtenidas presentan cierta asimetría en la calidad de los distintos materiales.

Entre éstos, los materiales porosos o heterogéneos, y aquellos sometidos a un proceso de cocción han permanecido entre los más dificultosos por su compleja geometría y los fenómenos lumínicos involucrados en su modelización.
Tal vez el caso más emblemático de los materiales cocidos es el pan, debido a su importancia en la vida cotidiana.
Como nota de color, un reconocido  científico del área, Alain Fournier, pronunció en 2001 la frase ``la Computación Gráfica todavía no ha sido capaz de renderizar de manera convincente una feta de pan''.
En los siguientes capítulos mostraremos que solo unos pocos intentos por abordar el problema fueron llevados a cabo pocos años después de la emisión de dicho comentario.

La geometría de estos materiales porosos ha sido estudiada muy superficialmente en computación gráfica, donde pocos trabajos abordaron el tema.
Por otro lado, estas geometrías han sido objeto de estudio sistemático en el campo de la ingeniería de los alimentos.
Sin embargo, debido a la compleja naturaleza de los fenómenos involucrados en la formación de estos materiales, los estudios resultan generalmente fenomenológicos y sólo proveen resultados estadísticos.
Este escenario, junto con el crecimiento cuasi-exponencial de poder de cómputo existente, gracias al desarrollo de las placas gráficas, proveen la base y la motivación de la presente tesis.

\section{Alcances y Objetivos}
Esta tesis propone el estudio e implementación de algoritmos tanto en CPU como GPU, que permitan un modelado y renderización adecuados y realistas de materiales porosos y cocidos.
Como material principal, la presente tesis tomará al {\em pan}, ya que el mismo se encuentra en la intersección de ambas familias, permitiendo derivar algoritmos en ambos sentidos.
Particularmente, se proponen algoritmos procedimentales (es decir, sin intervención del usuario) de generación de la geometría interna y externa (miga y corteza) del pan.
Además, se propone una representación adecuada, junto a un algoritmo de renderizado en GPU para las geometrías resultantes.
Finalmente, se validan los resultados utilizando técnicas multifractales.
Luego estos algoritmos se derivan a otros materiales.

\subsection{Aplicaciones}
Los algoritmos desarrollados poseen diversas aplicaciones en el campo del modelado procedimental, y en el renderizado de materiales.
Más específicamente, pueden mencionarse:

\begin{itemize}
\item {\bf Modelado de materiales con geometrías porosas microscópicas y macroscópicas}: las técnicas fractales desarrolladas capturan eficazmente variadas geometrías porosas. Es posible utilizar el algoritmo semi-automático de búsqueda de parámetros presentado, en la representación de otras geometrías macro-porosas. Estos resultados podrían generalizarse, y aplicarse en el desarrollo de técnicas de Monte-Carlo para la obtención de distribuciones estadísticas de radiancia de materiales con superficie microscópicamente porosa.
\item {\bf Renderización foto-realista}: como se explicará a lo largo de la tesis, la existencia de una geometría realista resulta fundamental en la obtención de métodos prácticos para su renderizado. Las geometrías resultantes de los algoritmos de modelado que serán presentados, serán expresadas en una representación que guiará la elección de métodos prácticos, posibilitando una visualización realista.
\item {\bf Visualización médica realista}: la utilización de volúmenes en la representación de los materiales permite su aplicación en contextos médicos, donde se cuenta con datos provenientes de tomografías computadas, las cuales presentan materiales porosos como huesos. Gracias a las técnicas desarrolladas, la visualización de esos materiales podría ser mejorada drásticamente, permitiendo un mejor diagnóstico en caso de enfermedades o lesiones.
\item {\bf Juegos serios}: los simuladores de escenarios realistas han sufrido una expansión dramática en la última década gracias al poder computacional actual \cite{Susi2007}. En ellos, una representación realista es indispensable para propósitos de entrenamiento, y por lo tanto, la generación de materiales basadas en primeros principios puede fortalecer en gran medida la sensación de inmersión necesaria en dichos simuladores. 
\item {\bf Extracción de características de materiales}: los métodos multifractales utilizados pueden ser aplicados en la discriminación de muestras reales de otros materiales, así como en la caracterización de los mismos. Estos estudios pueden devolver parámetros útiles en la generación procedimental de los mismos.
\item  {\bf Clasificación de imágenes de materiales}: el espectro multifractal demostró ser útil en la clasificación y caracterización de imágenes de miga de pan. Es posible que el mismo sea útil en la clasificación de imágenes de materiales similares, pudiendo utilizarse al método en cadenas de producción, o en la determinación de parámetros de calidad del material.
\end{itemize}

\section{Resultados Originales Presentados}
En la obtención de los mencionados objetivos, se tomaron en consideración sistemas dinámicos por un lado (en la generación de geometrías procedimentales), y el proceso de cocción del pan, y de manera más general, la cadena completa de pasos existente en la elaboración del mismo (leudado, cocción, etc.).
Debido a que la literatura de ingeniería de los alimentos presenta estudios muchas veces inconexos sobre los distintos pasos involucrados, se implementaron algoritmos que aproximan los pasos más preponderantes: leudado y cocción, dejando de lado otros procesos como el amasado, o la formación de la corteza.

En forma resumida, las contribuciones más preponderantes incluyen:
\begin{itemize}
\item un algoritmo fractal \cite{Mandelbrot1983} que induce una distribución de radios de burbujas consistente con el proceso de leudado de la masa cruda del pan,
\item un algoritmo que aproxima el proceso de cocción del pan \cite{Powathil2004},
\item un algoritmo que aproxima el efecto de levantamiento del pan durante la cocción, basado en las distribuciones de burbujas existentes en el leudado,
\item una técnica que permite asignar colores en la corteza, utilizando el espacio de color CIELab \cite{Hunter58}, indicando mayor o menor cocción en cada punto,
\item el estudio del espectro multifractal \cite{Xu2009} en la discriminación, y clasificación de imágenes de migas de pan,
\item un algoritmo procedimental que genera texturas volumétricas de geometrías porosas utilizando sistemas dinámicos, y,
\item un algoritmo que utiliza el espectro multifractal en la aproximación de una distribución de burbujas dada.
\end{itemize}

Los resultados de la investigación de la presente tesis doctoral fueron presentados en las siguientes publicaciones:

\subsection*{Revistas Internacionales}

\begin{itemize}
\item {\bf Procedural Bread Making}. Rodrigo Baravalle, Gustavo Patow y Claudio Delrieux. En {\it Computers \& Graphics}. {\bf 50}(2015):13-24.

doi:10.1016/j.cag.2015.05.003.

\item {\bf Multifractal Characterisation and Classification of Bread Crumb Digital Images}. Rodrigo Baravalle, Claudio Delrieux y Juan Carlos Gómez. En {\it EURASIP Journal on Image and Video Processing}. {\bf 2015}:9. Abril 2015. doi:10.1186/s13640-015-0063-8.

\item {\bf Real Time Realistic Modeling and Rendering of Porous Materials}. Rodrigo Baravalle, Leonardo Scandolo, Claudio Delrieux y Cristian García Bauza, en {\it Computer Animation and Virtual Worlds}, el cual está en proceso de revisión.
\end{itemize}


\subsection*{Revistas Nacionales}

\begin{itemize}
\item {\bf Modelado para la Renderización Foto-Realista de Pan}. Rodrigo Baravalle, Leonardo Scandolo, Claudio Delrieux y Cristian García Bauza. En {\it Mecánica Computacional} ({\bf 33})26:1695-1709. Bariloche, Argentina. Septiembre 2014.
\item {\bf Bread Crumb Classification Using Fractal and Multifractal Features}. Rodrigo Baravalle, Claudio Delrieux y Juan Carlos G\'omez. En {\it Mecánica Computacional} ({\bf 31})17:3013-3025. Salta, Argentina. Noviembre 2012.
\end{itemize}


\subsection*{Conferencias}
\begin{itemize}
\item {\bf Síntesis Procedimental de Materiales: resultados en el modelado de pan y materiales cocidos}. Rodrigo Baravalle, Claudio Delrieux y Juan Carlos G\'omez. En {\it Actas del XIV Workshop Investigadores en Ciencias de la Computación }, WICC. Posadas, Argentina, Abril 2012.
\item {\bf Síntesis de Texturas utilizando Sistemas de Partículas}. Rodrigo Baravalle, Claudio Delrieux y Juan Carlos G\'omez. En {\it Actas de la Cuarta Escuela en Ciencias de las Imágenes}, ECIMAG. Buenos Aires, Argentina, Agosto 2011.
\end{itemize}

\subsection*{Pósters y Resúmenes}
\begin{itemize}
\item {\bf Experiencias con Cython y PyOpenCL}. En {\it Actas de la Segunda Conferencia de Python en la Ciencia}, SciPyConAr. Puerto Madryn, Argentina. Octubre 2014.
\item {\bf Imfractal: Dimensiones Fractales con Python}.  En {\it Actas del Primer Conferencia de Python en la Ciencia}, SciPyConAr. Puerto Madryn, Argentina. Mayo 2013.
\item {\bf Clasificación de Muestras de Pan Utilizando Características Fractales}. Rodrigo Baravalle, Claudio Delrieux y Juan Carlos G\'omez. En {\it Actas de la V Escuela en Ciencias de las Imágenes}, ECIMAG. Santa Fe, Argentina, Julio 2012.
\end{itemize}


\section{Organización de la Tesis}
Los capítulos de la presente tesis se organizan de la siguiente manera: en el Capítulo 2 se introduce el estado del arte en el modelado y renderización de materiales en computación gráfica, explicando los casos más generales, y algunas derivaciones o casos particulares.
En el Capítulo 3 se introducen las dificultades presentes en materiales porosos, y una propuesta de modelado de los mismos, desde un punto de vista puramente procedimental, y otro con consideraciones físicas.
En el Capítulo 4 se presenta una propuesta de renderizado para la representación volumétrica inducida en el capítulo anterior, logrando imágenes foto-realistas y en tiempo real, gracias a la utilización de la placa gráfica.
En el Capítulo 5 se explican las conclusiones y los trabajos a futuro, los cuales se desprenden del trabajo de la presente tesis.
Finalmente, en el Apéndice A, se presenta una introducción a la programación de las placas gráficas actuales, tanto en aplicaciones gráficas, como en aplicaciones de propósito general.
