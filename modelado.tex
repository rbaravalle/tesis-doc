\chapter[Modelado de la Geometría de Miga de Pan]{Modelado Procedimental de Geometrías de Miga de Pan y Otros Materiales Porosos}

\section{Introducción y Motivación} % (Películas de Animación)
Como fue explicado en la motivación de esta tesis, determinados materiales recibieron menor atención en la literatura científica, debido entre otros factores, a las dificultades presentes	 en el modelado, a los excesivos costos computacionales, y/o a la necesidad de utilización del material en aplicaciones prácticas.
Con el paso de los a\~nos, la industria del cine y de los video juegos diseñó y aplicó procedimientos artísticos con el fin de subsanar estas deficiencias.
De esta forma, se busc\'o la mayor similitud posible en la apariencia, entre el material sintetizado y el real, sin considerar si el proceso provenía de una simulación física o era el resultado de un procedimiento artístico.

Entre los innumerables ejemplos que se pueden citar, uno particularmente relacionado al trabajo de esta tesis lo constituye la película Ratatouille \cite{Cho2007}.
La misma se desarrolla en un ambiente de cocina, y por lo tanto existen alimentos y materiales naturales que deben ser modelados para ser renderizados.
Debido a la inexistencia de modelos estándar de determinados materiales en la literatura de computación gráfica (comestibles), los artistas y programadores encargados de llevar a cabo la producción visual de la película debieron crear técnicas {\em ad-hoc} para lograr un renderizado realista de los diferentes materiales.
Sin embargo, el éxito visual obtenido no se vio acompañado de una liberación del código que producía dichas imágenes, por lo cual la técnica permaneció para uso privado de la compañía productora, resultando en una difícil reproducción de dichos materiales por parte de terceros.

En este capítulo se busca introducir soluciones a estas problemáticas: por un lado, la falta de modelos fenomenologicos documentados y por lo tanto reproducibles, sobre geometría de migas de panes y otros materiales porosos, y por otro, la utilización de modelos físicos basados en los procesos reales de formación de dichos materiales, los cuales provocan que el resultado sintético posea una mayor similitud visual y estructural a los materiales que se pretenden representar.

En este capítulo presentamos modelos procedimentales que permiten realizar simulaciones de la geometría de la miga de pan.
Como sub-producto presentamos un algoritmo que produce imágenes de texturas bidimensionales de materiales estocásticos como granito y mármol, y otros más estructurados como la madera.

\subsection{Consideraciones sobre el modelado de geometrías de migas de panes}
Como fue discutido en el capítulo anterior, la inmensa variedad de materiales, y sus complejas interacciones con la energía radiante que incide sobre los mismos, han sido un tópico central de investigación en las últimas décadas en Computación Gráfica.
El modelado de la naturaleza en física ha demostrado capturar de manera muy precisa fenómenos de transporte de la luz.
Las aproximaciones computacionales a estos modelos son una de las ramas más investigadas en el renderizado de materiales.
El modelado geométrico de materiales sigue una estrategia de investigación similar. Primero se construye un modelo físico del material, el cual luego se aproxima computacionalmente.

La elección del modelo geométrico depende de cada material, y de la escala de representación buscada.
Esta elección influencia fuertemente el algoritmo de renderizado. 

%Existen dos tipos principales de modelos para capturar la apariencia de materiales: superficies y volúmenes.
%Para esto existen dos ecuaciones principales.
%Por un lado, la ecuación del rendering \cite{Kajiya1986} captura la microgeometría de los materiales y su interacción con la luz en superficies.
%Por otro, la ecuación del rendering de volúmenes \cite{Kajiya1984} modela fenómenos volúmetricos como la transmitancia, la extinción, etc., debido a propiedades del medio (o material).

Si el material a representar presenta características homogéneas (metales, plásticos, y similares), típicamente se lo modela por medio de una superficie \cite{Neumann1999}.
Esta estrategia modela la superficie del material asumiendo distribuciones estadísticas en su microgeometría.
Para lograr capturar detalles mesoscópicos, por ejemplo en el caso de maderas o ladrillos \cite{Lefebvre2000}, una técnica usual consiste en precomputar dichas características en texturas, las cuales luego son mapeadas en la superficie.
La utilización de superficies constituye un enfoque menos costoso computacionalmente, pero por otro lado impone limitaciones al realismo que se puede alcanzar, sobre todo en aquellos detalles que son propios de la estructura a nivel mesoscópico y la interacción de la misma con la luz (por ejemplo luz que pasa a través de agujeros en una superficie).

Siguiendo este razonamiento, la miga del pan constituye un material extremadamente complejo de capturar, modelar y renderizar, debido a la geometría a distintos niveles (microscópico y mesoscópico) y su resultante interacción con la energía radiante.
De esta forma, la utilización de superficies no resulta adecuada, ya que la textura del pan debe su apariencia en gran medida a los fenómenos previamente explicados.
Si bien existen métodos en la literatura que utilizan superficies para representar características mesoscópicas de los materiales como funciones bidireccionales de texturas (bidirectional texture functions, \acrshort{BTF}) \cite{Tong2002}, las mismas no manejan adecuadamente la interacción lumínica antes mencionada.
Existe un intento en la literatura por suprimir esta falta \cite{Tong2005}.
El método produce imágenes realistas, pero existen muchas dificultades prácticas en la captura, el cómputo y el renderizado.
Estas limitaciones provocan que la utilización de dicho método no sea práctica por el momento.
Además, el método resulta inflexible dado que para cada imagen del material se requiere repetir todo el proceso (dos imágenes distintas del material requieren dos capturas).
Tampoco se pueden obtener cortes del mismo, dada la naturaleza de superficie del método.

Por otro lado, existen numerosas publicaciones que tratan el modelado de materiales mejor adaptados a una representación volumétrica (por ejemplo humo o nubes) \cite{Chentanez2011,Zhou2008}.
La utilización de volúmenes resulta costosa computacionalmente pero presenta varias ventajas.
Una de ellas radica en su independencia de mallas de triángulos, a diferencia de las superficies, evitando las desventajas mencionadas para las mismas.

Finalmente, existe un compromiso al momento de modelar materiales complejos y elegir una representación plausible.
Donde se requiere simplicidad y velocidad (posiblemente tiempo real), se utilizan generalmente superficies, en cambio, si el foto-realismo constituye el objetivo final, resulta más adecuado optar por una representación volumétrica.
En los últimos años esta elección se ve perturbada por la aparición de hardware paralelo de alta velocidad (placas gráficas o GPUs), ya que un diseño adecuado podría alcanzar imágenes foto-realistas utilizando volúmenes en tiempos interactivos.

En este capítulo proponemos una representación volumétrica para la geometría de la miga de pan.
Existe una metodología volumétrica que simula geometrías estocásticas \cite{Perlin1989}, utilizando funciones algebraicas para construir la geometría.
Sin embargo, la utilización de funciones algebraicas resulta impráctica en el caso de materiales porosos estocásticos, como el pan.

\section[Síntesis de Texturas de Materiales]{Un Framework para la Síntesis de Texturas de Materiales}

La textura característica de la miga de pan emerge como resultado de numerosos procesos.
El crecimiento de la levadura en la masa cruda del pan produce una de las principales transformaciones texturales.
Este proceso atañe la mayor responsabilidad en la formación de la geometría porosa macroscópica del material.
La observación detallada de binarizaciones de fotografías de cortes en la masa cruda, durante distintas etapas del proceso de leudado y cocción del pan \cite{Scanlon2001}, permite sentar las bases de un algoritmo fenomenológico que responde a este comportamiento.
En dichas imágenes se observa un crecimiento pseudo-aleatorio de burbujas atrapadas en la masa.
La textura final debe su apariencia a un aglomerado de burbujas interconectadas con distintos tamaños y formas.

La implementación de este método constituye el principal motivo de esta sección.
Si bien el objetivo del mismo radica en la producción de geometrías, como sub-producto surgen imágenes o geometrías que recuerdan apariencias típicas de otros materiales.
Al asignárseles colores, permiten obtener imágenes aceptables de los mismos.

Presentamos entonces un método procedimental de generación de texturas de materiales en dos dimensiones, basado en el crecimiento de partículas.
Se define una textura como una imagen. o arreglo bidimensional de múltiples usos que almacena información sobre un fenómeno determinado.
En este caso, una textura representará el material directamente, a nivel macroscópico (es decir una imagen del mismo).
La simulación de imágenes de materiales usualmente se presenta en publicaciones separadas en la literatura, dado que los procedimientos que producen distintos materiales se basan en metodologías dispares.

El algoritmo introducido será especializado en la siguiente sección en la obtención de geometrías de la miga de pan y otros materiales porosos.

\subsection{Sistemas de Partículas}
Un sistema de partículas está compuesto por abstracciones denominadas {\em partículas}, las cuales coexisten en un mismo ambiente.
En un sistema de part\'iculas \cite{Reeves1983} las part\'iculas nacen, se desarrollan y mueren, respetando reglas que les son impuestas por el sistema. Estos sistemas intentan modelar la din\'amica del fen\'omeno a trav\'es del tiempo, siendo uno de sus principales intereses mostrar una animaci\'on del mismo \cite{Gao2010, Bagar2010, Lentine2010}.

Se propone utilizar estos sistemas como generadores procedimentales de texturas. Un enfoque similar puede observarse en \cite{Kranidotis98}, aunque el mismo no fue desarrollado en profundidad.
Aquí se sigue una linea de investigaci\'on similar, presentando un sistema de part\'iculas que est\'a caracterizado por el crecimiento de las mismas en cada iteraci\'on, las cuales ocupan texels (elemento individual de una textura bidimensional, correspondiente a una posición $(u,v)$ dentro de la misma) asign\'andoles colores, pudi\'endose detener al mismo en una iteración arbitraria, obteni\'endose una textura.
Debido a la utilización de una selección aleatoria de los texels vecinos, donde crece cada partícula, las im\'agenes resultantes presentan apariencia estocástica, símilarmente a materiales naturales.
A trav\'es de la direcci\'on de crecimiento de las part\'iculas se producen distintos patrones visuales.

Es posible representar materiales cl\'asicos como madera, granito y m\'armol, pero tambi\'en otros como mosaicos, pinturas, vegetaci\'on, entre muchos otros.
Un ejemplo de generación procedimental de texturas puede observarse en \cite{Baravalle2010}.
La diferencia principal con el enfoque presentado en este trabajo radica en el mecanismo de generaci\'on, dado que con esta nueva propuesta un {\em proceso temporal} produce los resultados, pudiendo representarse todo el proceso de formaci\'on de un material.
Adem\'as, cada ejecución del algoritmo produce una textura diferente, utilizando los mismos par\'ametros, debido a la naturaleza aleatoria de la generaci\'on (aunque el uso de semillas permite repetir texturas en el caso de requerirse, por ejemplo para generar baldosas id\'enticas).

Se utiliza como plataforma de implementaci\'on a WebGL\footnote{\em http://www.khronos.org/webgl/}, recientemente propuesta por el Khronos Group (Open Standars for Media Authoring and Acceleration), dado que la misma otorga portabilidad al modelo (es posible ejecutarlo en una gran variedad de navegadores independientemente de la plataforma en la que se encuentre la computadora). 
La implementación desarrollada permite incluirlo en aplicaciones gr\'aficas.
Dada la intuitividad de los parámetros del mismo (direcciones de crecimiento, cantidad de partículas iniciales, etc.), los artistas pueden interactuar con el programa, buscando obtener imágenes de distintos materiales.

\subsubsection{Desarrollo}
Inicialmente el sistema consta de un conjunto de part\'iculas $P$
\begin{equation}
P = \{p_{1}, ... , p_{n}\}, n  \in \mathbb{N},
\end{equation}
uno o varios colores {\em RGB}, los cuales pueden tomar las part\'iculas.
\begin{equation}
cols = \{col_{1}, ... , col_{m} \}, m \in \mathbb{N},
\end{equation}
un conjunto de direcciones de crecimiento $D$,
\begin{equation}
D = \{d_{1}, ... , d_{k} \}, k \in \mathbb{N},
\end{equation}
y una grilla $B_{N\times N}, N \in \mathbb{N} $ de texels con colores RGB asociados (inicialmente $(R,G,B)=(0,0,0)$).

Cada elemento del conjunto $P$ posee las siguientes propiedades:
\begin{equation}
p_{i} = \{T_{i}, C_{i}, d_{i}, color_{i}\}, 1 \le i \le n,
\end{equation}
donde:

$T_{i} = \{t_{1}, ... , t_{n_{i}}\}$: conjunto de texels {\em ocupados} por la part\'icula en $B$,

$C_{i} = \{c_{1}, ... , c_{m_{i}}\}$: conjunto de texels {\em contorno} de la part\'icula en $B$,

$d_{i} \in D$: direcci\'on de crecimiento,

$color_{i}$: color {\em RGB} de la part\'icula, con ecuaci\'on \cite{Reeves1983}
\begin{equation}
color_{i} = col_{h} + rand * varcolor,
\label{eqColor}
\end{equation}

\noindent
y donde rand constituye un n\'umero pseudo-aleatorio uniformemente distribuido entre $-1$ y $1$, $varcolor$ un parámetro en los reales y $col_{h} \in cols$.
Cada $t \in T_{i}$ tiene asociado un color que var\'ia con respecto a $color_{i}$, de similar manera a la ecuaci\'on (\ref{eqColor}).

El {\em contorno} de una part\'icula determina los texels que la misma puede incorporar. 
En cada iteraci\'on, se elige aleatoriamente un texel en $C_{i}$, se lo quita del mismo y se lo incorpora a $T_{i}$.
Luego se actualiza el contorno de acuerdo a $d_{i}$ y al nuevo texel incorporado.
Se repite el proceso $\forall p_{i} \in P$, respetando las siguientes restricciones:
\begin{eqnarray}
\forall p_{i}, p_{j} \in P, t \in T_{i} \Rightarrow t \notin T_{j}, \\
\forall p_{i} \in P, t \in T_{i} \Rightarrow t \notin C_{i}.
\end{eqnarray}

Es decir, si una part\'icula posee un texel en $T_{i}$, ninguna otra lo posee (las part\'iculas son disjuntas), y si una part\'icula posee un texel en $T_{i}$, \'este no est\'a en su contorno.

Las part\'iculas pueden {\em nacer} y {\em morir}, lo cual representa su inclusi\'on o eliminaci\'on de $P$.
Al morir, la part\'icula deja asociados los colores de su conjunto $T_{i}$ en $B$, pudiendo utilizarse posteriormente esta informaci\'on.

%Podemos intentar aplicar estas técnicas al modelado procedimental de texturas. 
En la Fig. \ref{resultados} se observan algunas im\'agenes sintetizadas con esta metodología.
En la primera fila, la primera y cuarta textura muestran patrones con direcci\'on vertical, direcci\'on elegida para evolucionar las part\'iculas.
La \'ultima figura de la segunda fila, muestra part\'iculas que crecen en 3 direcciones posibles: aleatorio ($50\%$), vertical ($25\%$) y horizontal ($25\%$), lo cual demuestra que se posee un control preciso sobre las formas resultantes en las texturas.

En la Fig. \ref{sintesis} se observa un ejemplo de s\'intesis.
De izquierda a derecha, se observan texturas obtenidas en las iteraciones $10$, $150$, $250$, $500$ y $1000$.
En la Fig. \ref{teteras} se observan texturas sintetizadas aplicadas a un objeto mediante mapeo de texturas, mostrando una calidad aceptable para ser utilizadas en aplicaciones gr\'aficas.

La Fig. \ref{muerte} muestra el efecto de muerte de part\'iculas.
Las mismas liberan el espacio en $T_{i}$ dejando asociado su color en $B$, permitiendo que las dem\'as puedan crecer en dichos texels.
De esta forma, puede utilizarse posteriormente la informaci\'on que la misma dej\'o asociada.
En este caso los colores de la part\'icula que libera el espacio y la ocupante fueron mezclados, obteni\'endose im\'agenes que parecen haber sido ``pintadas'' por capas.
%En la Figura \ref{software} se observa el modelo corriendo en un navegador web.


\begin{figure}[t!]
\centering
\includegraphics[scale=0.18]{resultados}
\caption{Distintas texturas sintetizadas utilizando sistemas de partículas.}
\label{resultados}
\end{figure}

Se muestra un ejemplo de s\'intesis de una textura en la Figura \ref{sintesis}. Deben seleccionarse los siguientes par\'ametros:

\begin{itemize}
\item Par\'ametro {\em cantidad de part\'iculas iniciales}, con valor $100$.
\item Par\'ametro {\em nuevas particulas por iteraci\'on}, con valor $1$.
\item Par\'ametros {\em color 1 y 2 (RGB)}. con colores de ejemplo, uno con la componente verde mayor y otro con mayor componente azul.
\item Par\'ametros {\em direcciones}: aleatorio: $50\%$, diagonal $50\%$.
\item Par\'ametro {\em variaci\'on de color}, con valor 0.1, en una escala [0,1] (ver $varcolor$ en la secci\'on anterior).
\item Par\'ametro {\em variaci\'on de color por part\'icula}, con valor 0.1, en una escala $[0,1]$, el cual determina la variaci\'on del color por texel dentro de la part\'icula.
\end{itemize}

\begin{figure}[t!]
\centering
\includegraphics[scale=0.12]{sintesis}
\caption{Ejemplo de s\'intesis de una textura.}
\label{sintesis}
\end{figure}

\begin{figure}[t!]
\centering
\includegraphics[scale=0.14]{teteras}
\caption{Texturas generadas mapeadas en una tetera.}
\label{teteras}
\end{figure}


\begin{figure}[t!]
\centering
\includegraphics[scale=0.2]{muerte}
\caption{Efecto producido utilizando el concepto de muerte de las partículas.}
\label{muerte}
\end{figure}

Los resultados obtenidos muestran una flexibilidad útil para producir distintos materiales.
Como fue aclarado, la idea de utilizar sistemas de partículas para producir materiales será utilizada en la sección siguiente.
Para esto, se generalizarán los resultados a tres dimensiones.

\section{Modelado Procedimental de la Geometría de la Miga de Pan}


Utilizamos los sistemas de partículas previamente mencionados, en un cubo (tres dimensiones), pero reemplazando las direcciones de crecimiento por sistemas dinámicos \cite{Strogatz2001}, para evolucionar al sistema, produciendo patrones con formas geométricas naturales.
Este constituye un intento por simular el efecto de la levadura en la masa cruda del pan.
Estos procedimientos producen distribuciones de burbujas de distintos tamaños, resultando en un mecanismo controlable, que presenta propiedades estadísticas.

Si bien la metodología introducida no está basada en primeros principios, la misma constituye un paso previo que permite sentar las bases de un modelo inspirado físicamente, el cual se presentará en la siguiente sección.

\section{Sistemas Dinámicos como Modelo de Miga de Pan}
El propósito de este algoritmo constituye producir una geometría, la cual se renderizará posteriormente.
Por lo tanto, en lugar de devolver el color de una posici\'on espec\'ifica, el algoritmo genera un campo escalar compuesto de $0$s y $1$s ($0$ si la posici\'on contiene aire, $1$ si la misma contiene masa).
%Esta representaci\'on resulta adecuada para ser renderizada utilizando DVR.

Similarmente al caso de generación de texturas, el sistema consta de un conjunto de part\'iculas $P$,

\begin{equation}
  P = \{p_{1}, ... , p_{n}\}, n  \in \mathbb{N},
\end{equation}

\noindent una grilla $L_{N\times N \times N}, N \in \mathbb{N} $ (inicialmente $L_{xyz}=1$) de masa y aire como fue descripto previamente, y otra grilla $L^{2}_{N\times N \times N}$, (inicialmente $L^{2}_{xyz}=-1$) de posiciones donde cada celda almacena un único entero que indica qu\'e part\'icula adue\~na la misma ($i$ si el elemento de la grilla pertenece al contorno o interior de la part\'icula $i$).

Cada elemento en $P$ posee las siguientes propiedades:

\begin{equation}
  p_{i} = \{O_{i}, C_{i}\}, 1 \le i \le n,
\end{equation}

\noindent donde:

\begin{itemize}
\item $O_{i} = \{o_{1}, ... , o_{n_{i}}\}$: (Ocupadas) vector (conjunto) de posiciones ocupadas por la part\'icula en $L$.

\item $C_{i} = \{c_{1}, ... , c_{m_{i}}\}$: (Contorno) vector (conjunto) de posiciones representando el {\em contorno} de la part\'icula en $L$.
\end{itemize}

El vector $O$ representa las posiciones que ser\'an afectadas por la part\'icula, y el contorno $C$ se utiliza para asegurar que las part\'iculas se eviten entre s\'i.

El procedimiento se describe, simplificadamente, en el Algoritmo $1$. 

\begin{algorithm}[h!]
\caption{Algoritmo de modelado}
\begin{algorithmic}

\State{$t  = 0$}
\Comment{tiempo - iteración}
\State{$P  = []$}
\Comment{partículas}
\State{$L  = matriz(MxMxM).valores\_iniciales(1)$}
\Comment {Geometría - iniciada a 1 (masa)}
\State{$L^{2} = matriz(MxMxM).valores\_iniciales(-1)$}
\Comment{Dominio de cada partícula}

\For{$i \in [1,Cantidad\_Particulas]$}
    \Comment{Cada partícula toma una posición aleatoria en L}
    \State{$x \gets aleatorio, y \gets aleatorio()$}
    \State{$O[i] \gets [[x,y]]$}
    \State{$C[i] \gets []$}
    \For{$v \in vecindario(x,y)$}
        \State{$C[i].agregar(v)$}
    \EndFor
    \State{$P.agregar([O[i],C[i]])$}
\EndFor

\For {$t \in [0,tiempo\_max]$}
    \For {$i \in [1,Cantidad\_Particulas]$}
        \If {$vacio?~C[i]$}
            \State{morir()}
        \EndIf
        \For{$h \in C[i]$}
            \State{$C[i].eliminar(h)$}
            \Comment{la posición ya fue explorada}
            \State{// Si la posición o el contorno pertenece a otra partícula, elegir otra posición}
            \State{// borde\_libre chequea que el vecindario no este dominado por otra partícula}
            \If{$!(L^{2}[h] > 0 ~\&\&~ L^{2}[h] != i ~\&\&~ borde\_libre(separacion)$}
                \State{// Posición libre, ocupar}                
                \State{$L[h] \gets 0$} \Comment{masa -> aire}
                \State{$O[i].agregar(h)$}
                \State{$C[i].agregar(vecindario(h))$} 
                \State{$L^{2}.setear(vecindario(h),i)$} \Comment{Marcar posiciones en $L^{2}$ como $i$}
                \State{$L^{2}.setear(h,i)$}
                \Comment{turno de la particula $i+1$...}
            \EndIf
        \EndFor
    \EndFor
\EndFor
\end{algorithmic}
\end{algorithm}

Cuando $t = 0$, un conjunto de part\'iculas iniciales toman posiciones aleatorias en la grilla.
Para cada partícula, la posici\'on elegida resulta la primera posici\'on ocupada en $O$, adem\'as, el vecindario de $O$ se agrega a $C$.
Cada part\'icula evoluciona en un intento por extender sus posiciones ocupadas ($O$), marcando posiciones en $L$. Las posiciones se toman de $C$.
Cuando una part\'icula ocupa una posici\'on, la misma se elimina de $C$ y se agrega a $O$.
Luego el vecindario de esa posici\'on se agrega a $C$ (es decir las posiciones que rodean inmediatamente a la posici\'on tomada).
Las grillas se actualizan de la siguiente manera: $L$ se setea a $0$ en la posici\'on y $L^{2}$ se setea con el valor $i$ en las posiciones que se agregan a $C$. Las part\'iculas s\'olo pueden crecer si el valor encontrado en $L^{2}$ no pertenece a otra part\'icula.
El tama\~no del vecindario se par\'ametriza, definiendo la distancia entre part\'iculas.
Si el vector $C$ est\'a vac\'io, la part\'icula {\em muere} ya que no puede continuar creciendo.


El algoritmo puede ser terminado en cualquier $t$ deseado. El mismo puede finalizar su c\'omputo ante determinados eventos, por ejemplo, cuando todas las posiciones de $L^{2}$ fueron tomadas por part\'iculas, ya que no pueden realizarse progresos.

Variando el par\'ametro de distancia entre part\'iculas (separación) se obtienen distintas estructuras (ver Fig.~\ref{fg:sistdin1}). Las im\'agenes muestran ejemplos 2D (para mayor claridad) de crecimiento aleatorio de part\'iculas. La regi\'on blanca en las im\'agenes representa la masa restante luego del proceso.


\begin{figure*}[htb!]
  \centerline{\includegraphics[width=13cm]{sistdin1}}
  \caption[Efecto del parámetro separación entre partículas.]{Efecto del parámetro separación entre partículas. Izquierda: separaci\'on = 1, centro: separaci\'on = 2, derecha: separaci\'on = 4.}
  \label{fg:sistdin1}
\end{figure*}

Finalmente, el algoritmo devuelve la grilla $L$, la cual ser\'a renderizada posteriormente. En la siguiente secci\'on se explica la utilizaci\'on de sistemas din\'amicos en la evoluci\'on guiada de part\'iculas.

\subsection{Sistemas Din\'amicos}

Las ecuaciones diferenciales tienen como prop\'osito tratar con la dificultad (o imposibilidad) de hallar soluciones anal\'iticas en procesos din\'amicos.
En primer lugar, se define un modelo matem\'atico del fenómeno, del cual luego se obtienen 
las ecuaciones diferenciales asociadas. Las mismas se resuelven por medio de aproximaciones num\'ericas en cada instante de tiempo.

Los costos computacionales de estas soluciones dependen de la complejidad del fenómeno y el n\'umero de ecuaciones del sistema. En este trabajo proponemos utilizar una sub-\'area de ecuaciones diferenciales llamada ecuaciones diferenciales ordinarias (ODE). En esta representaci\'on, el tiempo se trata como la \'unica variable independiente.


De manera general, las ODEs se representan utilizando el siguiente sistema de ecuaciones:
\begin{equation} \label{eq:simple}  
  \begin{aligned}
    \dot{x_{1}} = f_{1}(x_{1},\ldots,x_{n}),\\
    \ldots\\
    \dot{x_{n}} = f_{n}(x_{1},\ldots,x_{n}),
  \end{aligned}
\end{equation}

\noindent donde $\dot{x_{i}}$ representa la derivada de $x_{i}$ con respecto
a $t$.
Las variables $x_{i}$ y las funciones $f_{i}$ se definen de manera diferente en cada modelo.
En este caso, cada variable representa una coordenada cartesiana en el espacio, $x_{1}$ es $x$, $x_{2}$ es $y$ y $x_{3}$ es $z$.
El conjunto de $f_{i}$ ser\'a definido tratando de capturar la estructura interna del pan. 
La siguiente secci\'on muestra cómo estos sistemas pueden describir la evoluci\'on de los sistemas de part\'iculas.

\subsection{Evoluci\'on de sistemas de part\'iculas utilizando sistemas din\'amicos}

La percepci\'on humana puede detectar patrones en la estructura de la miga de pan. Distintas observaciones pueden realizarse sobre la distribuci\'on de las burbujas en la misma (ver Fig.~\ref{fg:panreal}).
Primero, la forma de las burbujas cercanas a la corteza tiende a estirarse paralelamente a la misma, como resultado de la acci\'on de las elevadas temperaturas durante la cocci\'on de la masa.
Tambi\'en resulta evidente que la estructura completa posee la apariencia de un fluido con la forma de la corteza.


\begin{figure*}[htb!]
  \centerline{\includegraphics[width=13cm]{panreal}}
  \caption{Imágenes de cortes reales de pan}
  \label{fg:panreal}
\end{figure*}

Por otro lado, los sistemas din\'amicos previamente presentados producen formas naturales (ver Fig.~\ref{fg:sistdin2}).
En las im\'agenes pueden observarse c\'irculos y espirales, entre otras formas.
Las im\'agenes se obtuvieron dibujando trayectorias sobre un plano, siguiendo distintas ODEs.
Tres ODEs describen las din\'amicas presentes en las im\'agenes.
A modo de ejemplo, el siguiente conjunto de ecuaciones produce la imagen a la izquierda:

\begin{equation} \label{eq:simple}  
  \begin{aligned}
    \dot{x} &= x^{2}-y^{2}+1,\\
    \dot{y} &= 2xy+1.
  \end{aligned}
\end{equation}


\begin{figure*}[htb!]
  \centerline{\includegraphics[width=13cm]{sistdin2}}
  \caption{Sistemas dinámicos en el plano.}
  \label{fg:sistdin2}
\end{figure*}

En los ejemplos se eligen posiciones aleatorias y luego el sistema se resuelve por medio de un {\em solver} Runge-Kutta de cuarto orden, el cual permite conocer la direcci\'on a tomar en cada punto por la trayectoria (en las im\'agenes se computaron trayectorias hacia adelante y hacia atrás en el tiempo para lograr una visualizaci\'on más clara).
La imagen de la izquierda muestra un atractor y un repeledor claramente visibles.
El centro de la espiral m\'as a la izquierda constituye un atractor (a medida que $t$ avanza, las trayectorias convergen hacia el punto), mientras que el otro centro representa un repeledor.
Los atractores pueden no ser puntuales, como muestran las restantes dos im\'agenes.
La imagen de la derecha muestra atractores en forma de c\'irculos (las trayectorias ciclan por el c\'irculo).

Las part\'iculas producen distintos patrones al seguir las trayectorias definidas en el plano y el espacio.
Esto se logra resolviendo num\'ericamente el sistema din\'amico en la posici\'on agregada a la part\'icula, seleccionando como siguiente posici\'on de crecimiento aquella posici\'on del contorno que aproxima con mayor precisión la soluci\'on del sistema din\'amico (para esto, sólo se agrega al contorno esa posición).
El Algoritmo $2$ muestra cómo se modifica el algoritmo de modelado para incluir este comportamiento.

\begin{algorithm}[h!]
\caption{Modificación del algoritmo de modelado por medio de sistemas dinámicos}
\begin{algorithmic}
\State $L[h]\gets 0$ \Comment{masa $\rightarrow$ aire}
\State $O[i].agregar(h)$
\State $solucion \gets Runge\_Kutta(h)$
\Comment {Se calcula la siguiente posición}
\State $vec = vecindario(h)$
\State $mejor = abs(vec[0] - solucion)$
\State $elegida = h$
\State $vec.eliminar(h)$
\For {$w \in vec$}
\State{// Se calcula la posición del vecindario que mejor aproxima al sistema}
    \If {$abs(vec[w]-solucion) < mejor$}
        \State $mejor = abs(vec[w]-solucion)$
        \State $elegida = w$
    \EndIf
    \If {$aleatorio() > 1-aleatoriedad$} 
    \Comment {$0 <= aleatorio() <= 1$}
        \State $C[i].agregar(w)$
    \EndIf
\EndFor
\State{// Se agrega al vecindario sólo la posición que mejor aproxima la solución}
\State $C[i].agregar(elegida)$
\end{algorithmic}
\end{algorithm}

Las part\'iculas se deforman globalmente en un patr\'on visualmente similar a las trayectorias que produce el sistema (ver Fig.~\ref{fg:sistdin3}).
En las im\'agenes, de izquierda a derecha se decrementa la {\em aleatoriedad} de las trayectorias.
El par\'ametro aleatoriedad seteado a $0.1$ genera la imagen de la derecha, lo cual significa que las burbujas eligen como siguiente posición de crecimiento la que mejor se acopla al sistema con una probabilidad de $0.9$.
La probabilidad se define como $1-aleatoriedad$, donde $0 \leq aleatoriedad \leq 1$.
Las ecuaciones del sistema son las mismas que las de la imagen derecha mostrada en la Fig.~\ref{fg:sistdin2}.

\begin{figure*}[htb!]
  \centerline{\includegraphics[width=13cm]{sistdin3}}
  \caption[Sistemas din\'amicos aplicados en sistemas de part\'iculas]{Sistemas din\'amicos aplicados en sistemas de part\'iculas. Efecto del parámetro {\em aleatoriedad}. De izquierda a derecha, aleatoriedad: 0.3,0.2,0.1 respectivamente. }
  \label{fg:sistdin3}
\end{figure*}

\subsection{Resultados y Limitaciones}
Las Figs.~\ref{fg:crumb} y \ref{fg:results2} muestran imágenes de geometrías tridimensionales sintetizadas a través del procedimiento de partículas descrito, utilizando el método de renderizado del siguiente capítulo.
Las imágenes muestran un objeto cuyo material asemeja al de la miga del pan.

\begin{figure}
  \centerline{\includegraphics[width=13cm]{figures/crumb}}
  \caption{Miga de pan sintetizada, observada desde distintos ángulos.}
  \label{fg:crumb}
\end{figure}


\begin{figure}
  \centerline{\includegraphics[width=13cm]{figures/results2}}
  \caption{Miga de pan sintetizada, observada desde distintos ángulos (con corteza agregada).}
  \label{fg:results2}
\end{figure}

Estos patrones pueden ser utilizados adem\'as en otros materiales cocidos, variando el par\'ametro de aleatoriedad.
Distintos sistemas de ecuaciones pueden utilizarse para definir distintos patrones.

Si bien se pueden generar distintas geometrías de manera sencilla y prácticamente automática, todavía existen limitaciones de flexibilidad en los tipos de pan que se pueden sintetizar.
Existe una dependencia sobre un sistema dinámico para producir la forma global de la deformación, siendo el sistema difícilmente controlable.
Además, no siempre se adapta de manera sencilla a la forma exterior del material.
Debido a esto, en la siguiente sección presentamos un algoritmo que permite superar estas deficiencias.
El mismo continúa la idea de una modelización volumétrica del material, pero el mismo está basado en modelos físicos del proceso de fabricado del pan, extraído directamente de la literatura de ingeniería de los alimentos.

\section{Modelado del Pan desde su Proceso de Fabricación}
La obtención de geometrías realistas de la miga de pan requiere comprender su proceso físico de formación.

\begin{figure*}
\includegraphics[width=13cm]{figures/pipeline}
\caption{Procedimiento semi-automático para obtener pan sintético foto-realista}
\label{FigPipeline}
\end{figure*}

Utilizando las ideas de la sección anterior, proponemos otro modelo volumétrico de generación de geometrías de migas de pan.
En esta sección proponemos unificar, y diferenciar, los pasos claves presentes en el proceso de fabricación del pan (sobre todo, leudado y cocción).
El procedimiento busca utilizar el proceso físico de generación de geometrías de migas y cortezas de pan a partir de la masa original.
Estos procesos han sido vagamente tenidos en cuenta en la literatura, la cual ha atendido siempre a procesos artísticos más simples, dada la complejidad de los mismos.

En primer lugar, reveeremos el estado del arte del proceso de fabricación del pan.
% e introduciremos modelos matemáticos requeridos para comprender el proceso de cocción.
Luego presentaremos los procesos de modelado, que junto con el modelo de iluminación del siguiente capítulo, permiten obtener imágenes foto realistas de distintos tipos de pan.

\subsection{Literatura sobre el proceso de formación del pan}
Si bien existen algunos ejemplos en la literatura sobre modelado y renderizado de la estructura de migas de pan \cite{Tong2005,Xenakis2007}, los mismos ignoran casi por completo el proceso real de formación.
Algunos trabajos pioneros aplicaron modelos físicos de cocción a determinados tipos de pan para su renderizado, buscando obtener animaciones (por ejemplo \cite{Rodriguez-Arenas2011}), pero el modelado de la geometría de las burbujas de la miga del pan no fue tenido en cuenta.

Por otro lado, el modelado procedimental de panes resulta un tópico multidisciplinario de investigación.
La ingeniería de los alimentos lleva varias décadas de publicaciones en el área, tratando de desarrollar una mayor comprensión del proceso de formación del pan.
Esta rama de la ciencia muestra que el leudado determina fuertemente las características presentes en la miga de pan, particularmente las burbujas \cite{Babin2006}.
La interacción entre la levadura y algunos nutrientes presentes en la masa produce {\em $CO_{2}$}. 
El diámetro de las burbujas (asumiendo una burbuja esférica) y sus distribuciones espaciales muestran estructuras con características fractales, exhibiendo autosimilaridad estadística en diferentes escalas de medición.
Determinados trabajos han computado las dimensiones fractales de estas estructuras en determinados tipos de pan \cite{Gonzales2008}, sugiriendo distribuciones fractales uniformes.
El modelado de la cocción del pan es sujeto de numerosos trabajos \cite{Mondal2008}.

El modelado procedimental utilizando fractales también atendió las necesidades de otros variados materiales como montañas \cite{Prusinkiewicz1993}, cráteres lunares, y distribución de las burbujas en quesos \cite{Mandelbrot1983}. 
Adicionalmente, determinados modelos matemáticos complejos representan el comportamiento y el crecimiento de diversos fenómenos naturales.
En computación gráfica, estos modelos son unas de las fundaciones usadas para modelar agua y fluidos \cite{Stam1999,Fedkiw2001}.
Estos trabajos utilizan complejos modelos diferenciales de otras ramas de la ciencia y las aproximan con técnicas numéricas.
En años recientes, la tecnología GPGPU \cite{Owens2007} permitió la posibilidad de alcanzar tiempos reales o interactivos en el cómputo y renderizado de estos modelos numéricos.

A pesar de todos estos avances, el modelado y visualización foto-realista de panes y materiales porosos todavía presenta diversos retos.
Además de un modelo geométrico realista, el renderizado requiere que se represente de manera adecuada los fenómenos de transporte de la luz, incluyendo auto-oclusión, auto-sombreado, transmitancia, translucencia, transparencia, entre otras.
Solamente unas pocas publicaciones proponen atacar ambos aspectos, pero utilizando consideraciones artísticas \cite{Xenakis2007}.
Además, estos autores no liberaron suficientes detalles del modelado y renderización ya que existen derechos de propiedad intelectual, limitando enormemente la reproducción de dichas imágenes.

Por otro lado, la comunidad artística usualmente produce imágenes realistas de pan utilizando fotografías de los mismos, y definiendo geometrías a partir de ellas, junto a la utilización de materiales translúcidos\footnote{http://www.blenderguru.com/tutorials/how-to-create-realistic-bread} y otras consideraciones {\em ad-hoc}\footnote{http://design.tutsplus.com/tutorials/create-a-realistic-loaf-of-bread-in-photoshop--psd-10555}.
Si bien los resultados obtenidos son buenos, los procesos son tediosos y demandan horas de trabajo.
Además, la obtención de una nueva imagen requiere repetir todo el proceso desde el principio, lo que torna al mismo muy poco práctico.
Existen numerosas desventajas además de esta.
Por ejemplo, no se contempla la obtención de cortes arbitrarios del material resultante.


\subsection{Visión Global del Proceso}
La Fig.~\ref{FigPipeline} muestra el proceso completo de formación de geometrías sintéticas de pan.
Todos los pasos se aplican sobre campos escalares.
Los usuarios pueden proveer el sistema con un modelo de tres dimensiones de su preferencia, o dejar que el sistema provea un pan de forma estándar ({\em croissants}, {\em baguettes}, etc.).

La geometría introducida se voxeliza para proceder a las siguientes etapas.
En la simulación del proceso de leudado, el usuario puede parametrizar la textura del pan (cantidad y tamaño de las burbujas y su distribución), o nuevamente dejar que el sistema provea parámetros estándar que producen tipos de pan conocidos.
La masa cruda será intersectada con la geometría voxelizada para obtener un campo escalar, el cual tiene la forma externa que provee el usuario (o el sistema), con el interior de la misma compuesta de las burbujas procedentes de los parámetros establecidos (ver Sección~\ref{breadprov}).
Luego, se computa un modelo de cocción específico \cite{Powathil2004} para deformar el campo escalar (y las burbujas de éste), de acuerdo a los efectos que produce la cocción en el proceso de formación del pan.
El último paso aplica renderizado directo de volúmenes (direct volume rendering, DVR) \cite{Kruger2003} al campo escalar resultado de la cocción, obteniendo imágenes realistas del pan resultante.

%====================================================================

\subsection{Voxelización de la Geometría}

El modelo permite generar panes con geometrías arbitrarias, supliendo un modelo de triángulos en algún formato estándar.
El modelo consta de una serie de pasos. El primer paso de la secuencia voxeliza un modelo provisto por el usuario con la utilidad de código abierto {\tt binvox} \footnote{http://www.cs.princeton.edu/$\sim$min/binvox/} \cite{Nooruddin2003}.
La voxelización genera un campo escalar binario, con $1$ representando que el voxel dado está dentro de la geometría, y $0$ significando que el voxel está fuera de la geometría.
El paso de leudado (siguiente subsección) genera la textura del material que se ubica dentro de esta geometría voxelizada.
Esto permite generar panes con formas arbitrarias, tales como {\em baguettes}, {\em croissants}, pan {\em lactal}, u otros menos comunes (una tetera o un conejo).

Un sub-producto importante de la matriz que representa la geometría está dado por un campo escalar secundario que se obtiene por medio de una transformación de distancia.
La transformación de distancia genera una matriz $n$-dimensional con las mismas dimensiones que la matriz que se transforma \cite{osh03}.
La transformación computa, para cada entrada distinta de cero en la matriz, la distancia más cercana a una entrada nula de dicha matriz.
Dada una matríz $n$-dimensional $M$, el algoritmo genera una matriz a valores reales $DF_{M}$ con las mismas dimensiones que $M$,


$$  DF_{M}[i] = \min \bigg\{ \delta(i,j): M[j] = 0 \bigg\},$$

%\begin{align*}
%M'[i] &= min(d(i,j)), M[i] != 0, M[j] = 0\\
%M'[i] &= 0, M[i] = 0,
%\end{align*}

\noindent
donde $\delta(i,j)$ podría ser la distancia de Manhattan o la distancia Euclideana en un espacio $n$ dimensional, e $i$ y $j$ son vectores de $n$ dimensiones.
Las entradas de la matriz lejanas a los bordes del objeto toman valores mayores que aquellas cercanas a los mismos.
De esta manera se obtiene un mapa de la distancia de cada voxel a las superficies tridimensionales del objeto.
Este campo escalar de distancias será requerido en etapas posteriores del proceso.
La Fig.\ref{fg:distance} muestra un ejemplo de un mapa de distancias.

\begin{figure*}
\includegraphics[width=13cm]{figures/distance}
\caption[Mapa de distancias de una matriz bidimensional]{Mapa de distancias de una matriz bidimensional. A la izquierda se observa la matriz original, y a la derecha su mapa de distancias. Se observa que en los bordes el valor del mapa resulta mínimo, aumentando hacia el centro.}
\label{fg:distance}
\end{figure*}


\subsection{Simulación del Leudado}
\label{breadprov}
Como fue discutido, la textura observable en la masa cruda del pan está compuesta de burbujas, cuya distribución exacta resulta de procesos complejos, entre ellos, reacciones químicas, y deformaciones físicas de la masa.
El paso de leudado consta en primera instancia del crecimiento libre de las burbujas, producido por la interacción de seres vivos (levadura) con la masa, la cual libera $CO_{2}$, resultando éste atrapado dentro de la masa.
Luego, el cocinero interviene la masa deformándola de diversas maneras.
En este paso, la masa se deja reposar, efectuándose un segundo leudado.
Finalmente el paso de cocción produce la textura y distribución de burbujas finales.

Debido a la extrema complejidad de los pasos involucrados, las distribuciones exactas de burbujas carecen aún de explicación, por lo tanto existen estudios fenomenológicos sobre estas texturas, las cuales utilizan tomografía de rayos X y extracción de características sobre las mismas \cite{Babin2006,Gonzales2008,VanDyck2014}.
Buscando obtener una variabilidad similar, en esta tesis generamos distribuciones de burbujas con un modelo basado en fractales, inspirado en las distribuciones de burbujas presentes en el queso, y la textura inducida por los cráteres lunares, propuesto en \cite{Mandelbrot1983}.
Luego validamos los mismos utilizando el método multifractal denominado {\em Sandbox} (caja de arena, ya que realiza cálculos sobre una distribución de puntos que asemeja a arena dispersa de manera aleatoria en la geometría).
Esto será explicado en profundidad en el siguiente capítulo (validación).

La textura de la masa se genera procedimentalmente en un campo escalar separado, definido en una matriz de tres dimensiones.
Las dimensiones son las mismas de la matriz de geometría definida en la subsección anterior, donde cada voxel se inicializa a $1$ (significando que el material aún no tiene burbujas).
El proceso comienza sustrayendo esferas de radio $r_{min}$ posicionadas arbitrariamente en el campo escalar (esto se logra seteando a $0$ las celdas respectivas).
Luego, substraemos esferas de radio mayor, nuevamente en posiciones arbitrarias, hasta un radio máximo $r_{max}$.
La relación entre el número de esferas $N_{s}$ a ser sustraídas en cada paso, y sus respectivos radios $r$, está dada por la ley fractal


\begin{equation*}
N_{s} = \frac{k}{r^{d}},
\end{equation*}

donde $d$ constituye el exponente fractal que modela la ocurrencia de esferas en relación con su radio, y $k$ controla la cantidad de esferas para cada radio.
Con estos simples parámetros basta para modelar una gran cantidad de texturas en general, particularmente pan.

La Fig.~\ref{FigProving} muestra un ejemplo de un corte en dos dimensiones de este modelo.
Si bien las burbujas esféricas resultantes no son completamente realísticas debido a su forma, los resultados muestran un notorio parecido en tamaños y distribución de tamaños con binarizaciones de cortes reales de masas leudadas (ver \cite{Babin2006}).
Durante la cocción, el campo escalar resultante será sometido a deformaciones geométricas. Consecuentemente, la textura final se ajustará aún más a burbujas de panes reales.
Finalmente, en la sección de validación se buscarán parámetros de generación que ajusten diferentes panes reales. 

\begin{figure}
\center
\includegraphics[width=7cm]{figures/bubbles}
\caption{Simulación Fractal del Leudado de Pan.}
\label{FigProving}
\end{figure}

Luego de la voxelización y la simulación de leudado, las burbujas deben estar presentes sólo en el interior de la geometría.
Para esto, intersectamos ambos campos escalares utilizando un simple producto punto a punto, es decir, una máscara,

\begin{equation*}
P_{2} = P_{1} * G,
\end{equation*}
%
donde $P_{1}$ nota el campo $3D$ (matriz) que contiene las burbujas del leudado, y $G$ el campo escalar que representa a la geometría voxelizada.

La Fig.~\ref{fg:intersectProblem} muestra una versión renderizada de esta intersección.
Si bien los resultados pueden parecer realistas, en panes reales no resulta habitual la visualización de burbujas en la superficie, debido al leudado y cocción reales.
Para solucionar este inconveniente, utilizamos el campo escalar de distancias que computamos previamente, para deshabilitar la generación de burbujas en zonas cercanas a los bordes de la geometría, de acuerdo a cierto valor umbral, el cual será un parámetro.
%El campo escalar binario resultante se computa de la siguiente manera,


\begin{figure*}
\includegraphics[width=13cm]{figures/intersectProblem}
\caption[Intersección simple de la geometría y de las burbujas provenientes del leudado]{Intersección simple de la geometría y de las burbujas provenientes del leudado. A diferencia de masas reales, pueden verse burbujas en la superficie de la masa.}
\label{fg:intersectProblem}
\end{figure*}


Las Figs.~\ref{fg:proving} y \ref{fg:provingBunny} muestran resultados de limitar las interacciones de burbujas a la región interior de la geometría original, donde se observa que no existen burbujas en la superficie de la masa, como ocurre en masas reales.
Las imágenes muestran que el método definido permite la producción de imágenes realistas de panes crudos con formas arbitrarias.
Cabe remarcar que el modelo resulta suficientemente flexible, permitiendo renderizar el material en cualquier etapa de su proceso de fabricación, además de permitir la posibilidad de realizar cortes arbitrarios sobre el mismo.

\begin{figure*}
\includegraphics[width=13cm]{figures/prebakebread}
\caption{Pan sintético luego del leudado, y antes de la cocción.}
\label{fg:proving}
\end{figure*}

\begin{figure*}
\includegraphics[width=13cm]{figures/prebakebunny}
\caption{Conejo de pan sintético, luego del leudado, y antes de la cocción.}
\label{fg:provingBunny}
\end{figure*}

%====================================================================
\subsection{Cocción}
%To adequately simulate the bread baking process, mathematical models should be able to accurately model the heat and mass transfer in dough.
Los modelos matemáticos del proceso de cocción del pan deben ser capaces de modelar correctamente las transferencias de calor y masa en el mismo.
Los modelos físicos más completos del proceso producen soluciones precisas a la distribución de temperaturas en la masa durante la cocción.
Sin embargo, estos modelos son usualmente muy complejos para implementar y son muy costosos computacionalmente.
Además, sólo los modelos más complejos toman en consideración la formación de la corteza del pan.
La corteza, por otro lado, aún carece de una definición formal~\cite{Vanin2009}.
Por lo tanto, los modelos antes citados incluyen sólo una simplificación de la formación de la misma, muy lejos de lo que ocurre en la cocción real.

Los resultados más relevantes en la literatura sugieren que un modelo unidimensional de la cocción puede ser suficiente para la mayoría de los casos prácticos.
Por ejemplo, Purlis~\cite{Purlis2011} modela la cocción del pan como un fenómeno unidimensional, representando la geometría como un cilindro infinito.
Otros trabajos asumen sólo una coordenada radial, resultando también en modelos $1D$~\cite{Powathil2004, Thorvaldsson1999}.
Estos trabajos muestran que utilizar una representación de una única dimensión produce resultados casi idénticos a aquellos que se obtienen utilizando un mayor número de dimensiones.

La simulación numérica que implementamos está basada en el esquema de diferencias finitas propuesto por Powathil~\cite{Powathil2004}, y Thorvaldsson y Janestead~\cite{Thorvaldsson1999}. 
El sistema presentado consiste de un conjunto de tres ecuaciones acopladas que describen transferencia de calor, difusión de vapor de agua y difusión de agua líquida.
En nuestro algoritmo, sólo utilizamos las temperaturas ($T$) como entrada para las siguientes etapas de la formación del pan.
La Ec.~\ref{Eq:heat} modela transferencia de calor en la masa del pan, tomando en consideración el balance de energía y evaporación del agua debido a la temperatura~\cite{Thorvaldsson1999},
%
\begin{equation}
\label{Eq:heat}
\frac{\partial T}{\partial t} = \frac{1}{\rho C_{p}} \frac{\partial}{\partial x} \left ( k \frac{\partial T}{\partial x} \right ) + \frac{\lambda}{C_{p}} \frac{\partial W}{\partial t}+\frac{\lambda W}{ \rho C_{p} }\frac{\partial \rho}{\partial t},
\end{equation}
%
donde $T$ representa la temperatura, $x$ es la coordenada radial, $t$ es el tiempo, $C_{p}$ el calor específico, $\rho$ la densidad, $k$ la conductividad térmica, $\lambda$ es el calor latente de la evaporación de agua, y $W(x,t)$ es el contenido de agua líquida. 
Las condiciones iniciales
%
\begin{align*}
T(x,0) &= T_{0}(x), 0\le x \le L/2,
\end{align*}
y las condiciones de borde (continuidad y suavidad) definen el modelo,
\begin{align*}
\left ( \frac{\partial T}{\partial x} \right )_{x=L/2} &= 0 , t > 0 \\
-k \left ( \frac{\partial T}{\partial x} \right )_{x=0} &= h_{r}(T_{r}-T_{s}) + h_{c}(T_{air}-T_{s}) - \lambda \rho D_{w} \left (\frac{\partial W}{\partial x} \right )_{x=0}
\end{align*}
%
donde $h_{r}$ y $h_{c}$ son sub-términos del coeficiente de transferencia de calor ($h = h_{r}+h_{c}$), $T_{air}$, $T_{s}$, $T_{r}$ son temperaturas en el aire circundante, en la superficie del pan, y en la fuente de radiación, respectivamente, $L$ es el diámetro del pan ($x = L/2$ se ubica en el centro del pan y $x = 0$ en el contorno del mismo), $D_{w}$ representa la difusividad del agua líquida, y $T_{0}$ la temperatura inicial. 
Las temperaturas están expresadas en Kelvin ($K$). 
El modelo consta de ecuaciones similares para la difusión del vapor de agua ($W$) y difusión de agua líquida ($V$).
Otros detalles del modelo pueden ser consultados en \cite{Thorvaldsson1999}.
En nuestro trabajo, utilizamos este modelo para obtener un mapa de temperaturas al final del proceso de cocción.
Estas temperaturas afectarán posteriormente las formas y tamaños de las burbujas.

La simulación de la cocción define el horno a una temperatura típica (por defecto utilizamos $210^{\circ}C$) y discretiza el tiempo en intervalos $\Delta t = 30s$.
De la simulación se obtiene un arreglo unidimensional $Temp$, de tamaño $N_{grid}$, compuesto por valores de temperatura.
Por estabilidad numérica, utilizamos $N_{grid}=32$ (como en el trabajo original) e interpolamos las temperaturas para obtener mayores resoluciones ($N_{int}$). 
%Each value represents a dough position after $M$ baking time steps. 

El vector obtenido presenta temperaturas decrecientes de $R = 0$ a $R = L/2$, ya que el centro y el borde del pan poseen la menor y la mayor temperatura, respectivamente, luego de la cocción (el calor fluye desde la corteza hacia el centro del pan). 
Si utilizamos el campo escalar de distancias computado previamente, en conjunto con el vector de temperaturas, podemos mapear distancias a temperaturas.
Esto permite definir temperaturas en el campo escalar de forma compatible a una simulación $3D$.
Basados en estos razonamientos, mapeamos el vector de temperaturas $Temp$ en $3D$ de la siguiente manera,

\begin{equation*}
\displaystyle R[(x,y,z)] = Temp[ round( DF_{G}[(x,y,z)] ) ], 
\end{equation*}
%
donde $R$ es el campo escalar resultante del mapeo, y $x$, $y$ y $z$ son coordenadas $3D$ en $R$. Cuando $DF[(x,y,z)] = 0$, la posición actual $3D$ se encuentra en la superficie de la geometría, y $Temp[0]$ se mapea $R[(x,y,z)]$.
Cuando $DF[(x,y,z)] > 0$, mapeamos una temperatura menor a $R$, ya que la posición actual se encuentra en el interior de la geometría.
La Fig.~\ref{fg:baking} muestra cortes del resultado de este mapeo ($R$), usando temperaturas resultantes de un tiempo arbitrario de cocción, en los tres planos Cartesianos. 
Las imágenes son casi idénticas a lo que se obtendría con una simulación $3D$~\cite{Purlis2010}.

\begin{figure*}
\includegraphics[width=13cm]{figures/tempsbunny}
\caption[Temperaturas unidimensionales mapeadas a un campo escalar tridimensional con forma de conejo]{Temperaturas unidimensionales mapeadas a un campo escalar tridimensional con forma de conejo. Las imágenes muestran que la distribución de temperaturas arroja resultados similares a una simulación en tres dimensiones.}
\label{fg:baking}
\end{figure*}

Luego de cierto número de pasos de simulación ($t=20$ en nuestro caso), computamos el campo vectorial del gradiente de $R$ \cite{Gonzalez2006}, $g$, y luego suavizamos el mismo utilizando un kernel Gaussiano.
Finalmente, utilizamos las versiones suavizadas para deformar el campo escalar de la siguiente manera,

\begin{align*}
\displaystyle
u &= x+p*g'_{x}[x,y,z],\\
v &= y+p*g'_{y}[x,y,z],\\
w &= z+p*g'_{z}[x,y,z]
\end{align*}
donde $(u,v,w)$ son las coordenadas en el campo escalar resultante, ($x,y,z$) son las coordenadas originales, $p$ es un parámetro real positivo que denota la intensidad con la que se deforma el campo (controla el efecto neto de la cocción en las burbujas), y las $g'$s son las versiones suavizadas del gradiente original $g$.
%Fig.~\ref{FigBakingVectorField} also shows the computed gradient, where using $[-g_{x},-g_{y}]$ produces a similar vector but in the clockwise direction that can be applied with similar effects on the bubbles. 
La Fig.~\ref{fg:bakedbubbles} muestra cortes $2D$ de campos escalares luego de aplicado el proceso de cocción a diferentes formas.

\begin{figure*}
\includegraphics[width=13cm]{figures/bakedbubbles}
\caption{Cortes bidimensionales con burbujas deformadas por el proceso de cocción en diferentes tipos de panes.}
\label{fg:bakedbubbles}
\end{figure*}

El parámetro $p$ puede ser utilizado para sintetizar diferentes apariencias de miga de pan, desde masa cruda hasta burbujas con mucha deformación (ver Fig.~\ref{fg:parameterp}).
Cuando incrementamos $p$, forzamos a las burbujas a seguir más ajustadamente la forma exterior del pan. 
La imagen también muestra que el método deforma más pronunciadamente las burbujas más cercanas a la corteza.

\begin{figure*}
\includegraphics[width=13cm]{figures/parameterp}
\caption[Efecto del parámetro $p$ en la simulación de cocción]{Efecto del parámetro $p$: de izquierda a derecha $p$ es $0$, $5$, $10$,$15$, y $25$, respectivamente.}
\label{fg:parameterp}
\end{figure*}


\subsection{Formación de la Corteza}
El modelo de cocción introducido no incluye detalles precisos de la formación de la corteza.
En estas ecuaciones, esta se asume como un sub-producto que se produce en la superficie del material, a determinada temperatura, pero esta simplificación constituye sólo una cruda aproximación al fenómeno en el proceso real de cocción.
La literatura de ingeniería de los alimentos define la corteza como una interfase que emerge entre la masa y el aire.
Entre sus principales características se encuentra la formación de un material más denso en la superficie, y una diferenciación de color.
Sin embargo, los detalles precisos acerca de cómo esto ocurre se desconocen, por lo tanto, la apariencia exacta de la misma, está lejos de ser comprendida en los modelos presentes actualmente.

Debido a esto, y basados en la aproximación inicial asumida, utilizamos el campo escalar de distancias computado previamente para definir una región de corteza.
Esta elección está fundada en el hecho de que la corteza está mayormente determinada por un frente de evaporación que resulta de la temperatura \cite{Jefferson2007}.
En otras palabras, obtenemos las posiciones de la corteza utilizando el campo escalar mencionado, ya que hemos definido una relación entre la temperatura y la distancia a la superficie.
Además, obtenemos diferentes apariencias de pan utilizando un parámetro de distancia que determina diferentes espesores de corteza.
Por completitud presentaremos además otro método, basado en morfología matemática, para calcular la misma.

\subsection{Determinación de la corteza utilizando morfología matemática}
Debido a que resulta difícil o imposible definir funciones matemáticas algebraicas que se adapten a cualquier forma, proponemos un método automático para determinar regiones exteriores de volúmenes utilizando el formalismo conocido como morfología matemática \cite{Gonzalez2001}.

Con esto se pretende obtener una región que se ajuste a los bordes de un campo escalar.
Para esto se utilizarán técnicas simples conocidas como apertura, cierre, erosión y dilatación.
Las operaciones las definiremos en campos escalares tridimensionales.

Un elemento estructurante $E$ denota un campo escalar binario que representa una forma particular (cubo, esfera, etc.).
Los elementos estructurantes son utilizados en la definición de las operaciones antes mencionadas.

La operación de erosión se utiliza para reducir el tamaño total del campo escalar.
La erosión de un campo escalar $A$ utilizando un elemento estructurante $B$ se define de la siguiente manera,

\begin{equation}
A \ominus B = \{z\in \mathbb{Z}^3 | B_{z} \subseteq A\},
\end{equation}

\noindent donde $B_{z}$ es la traslación de $B$ por el vector tridimensional $z$.

La dilatación aumenta el tamaño total del campo escalar, preservando en cierta manera la forma de sus bordes.
La dilatación de un campo escalar $A$ utilizando el elemento estructurante $E$ se define como sigue,

\begin{equation}
A  \oplus E = \bigcup_{e\in E} A_e.
\end{equation}

La operación de cierre se emplea para {\em cerrar agujeros} en el campo escalar.
El cierre de un campo $B$ utilizando un elemento estructurante $E$ se define como una dilatación seguida de una erosión,

\begin{equation}
B \bullet E = (B \oplus E) \ominus E,
\end{equation}

\noindent ver \cite{Gonzalez2001} para obtener mayores detalles.

Para obtener la región externa de la geometría representada en el campo escalar, se necesitan eliminar los agujeros internos que representan las burbujas.
Para esto, el primer paso produce un cierre $c$ del campo escalar utilizando un cubo (elemento estructurante de radio $r$), causando el cierre de las burbujas internas.
Llamaremos $d$ a la dilatación de $c$ con un elemento esférico de radio $r_{2}$ ($d = c \oplus E_{r_{2}}$) y $e$ a la erosión de $d$ utilizando un elemento estructurante esférico con un radio ligeramente menor ($r_{3}$), $e = d \oplus E_{r_{3}}$.
La diferencia entre $d$ y $e$ ($d-e$) se encuentra en el borde del campo escalar, por lo cual puede utilizarse como corteza.
Para eliminar posibles imperfecciones en los pasos anteriores, se suaviza el campo escalar por medio de un cierre de $d-e$ y una dilatación del resultado, obteniendo la corteza final.


La Fig.~\ref{fg:crusts} muestra ejemplos de campos escalares con cortezas agregadas utilizando este técnica. Los resultados se adaptan naturalmente a cualquier campo escalar, inclusive si existen agujeros grandes.
El resultado cubre completamente el campo escalar, por lo cual para poder observar la parte interna del material deben realizarse cortes en el mismo.

\begin{figure}
  \centerline{\includegraphics[width=13cm]{figures/crusts}}
  \caption[Determinación de la corteza del pan utilizando morfología matemática]{Determinación de la corteza del pan utilizando morfología matemática. La imagen de la derecha muestra que el método se comporta adecuadamente incluso en casos donde existen grandes agujeros en el campo escalar.}
  \label{fg:crusts}
\end{figure}

\subsection{Tiempos de cómputo}
En la implementación utilizamos los lenguajes Python\footnote{python.org} y Cython\footnote{cython.org}.
En la Tabla~\ref{tab:computingtimes} se muestran tiempos de cómputo típicos para los distintos pasos en la simulación de la formación del pan, detallados en secciones anteriores. %In addition, rendering times with DVR are in the order of $~300$ ms (milliseconds), achieving interactive framerates.

\begin{table}[h!]
       % Give a unique label
% For LaTeX tables use
\begin{tabular}{lllll}
\hline\noalign{\smallskip}
Resolución del Campo Escalar & $256^{3}$ & $384^{3}$  & $512^{3}$ \\
\noalign{\smallskip}\hline\noalign{\smallskip}
Leudado & 0.28 & 0.97 & 2.29 \\
Intersección & 8 & 10.81 & 14.97 \\
Transformación de Distancia & 7 & 23.73 & 56 \\
Cocción & 19.92 & 51.15 & 117.27 \\
\hline\noalign{\smallskip}
Total & 35.20 & 86.66 & 190.53 \\
%Rendering & 1s & 4s & 0\% \\
\noalign{\smallskip}\hline
\end{tabular}
\caption{Tiempos de cómputo típicos en de las distintas etapas de la fabricación sintética de pan, expresados en segundos.}
\label{tab:computingtimes}
\end{table}

Es importante mencionar que la mayor parte del tiempo de cómputo en la cocción está dedicado al cómputo del gradiente, y no al proceso de cocción en sí.
Debido a los costos computacionales en resoluciones superiores, el usuario puede generar vistas previas de los panes utilizando campos escalares con menores resoluciones.


\subsection{Resultados}
En esta sección presentamos imágenes de geometrías de pan obtenidas utilizando nuestro modelo, renderizadas con el método presentado en el siguiente capítulo.

Las Figs.~\ref{fg:renders}, \ref{fg:renders2}, y ~\ref{fg:renders3} muestran imágenes renderizadas de panes obtenidos con el método presentado en esta sección.
Las Figs.~\ref{fg:renders} y ~\ref{fg:renders2} marcan las diferencias en realismo entre resoluciones del campo escalar de ($256^{3}$) y  ($512^{3}$) respectivamente.
La Fig.~\ref{fg:renders3} muestra versiones de alta resolución en el campo escalar de otras geometrías, incluyendo panes con formas de conejos.
La Fig.~\ref{fg:croissant} muestra un pan con forma de {\em croissant} en el cual se han hechos distintos cortes, los cuales demuestran que el método permite obtener imágenes de partes arbitrarias del material.
Las Figs.~\ref{fg:bigalveoli} y ~\ref{fg:bakedbunny} muestran un pan típico y uno con forma de conejo, mostrando burbujas grandes y las burbujas deformadas como resultado del proceso de cocción.
Además se modificó el color de la miga.

Los colores de corteza y miga son parámetros definibles por el usuario. Los cortes en la miga son fácilmente producidos anulando las regiones deseadas en el campo escalar luego de la cocción, y antes del renderizado.

Respecto de la sección anterior, puede observarse una forma más suave en las burbujas resultantes, junto con la capacidad de modelar tipos arbitrarios de panes.
Además, el proceso está basado en la secuencia real de pasos de la fabricación del pan, permitiendo la obtención de imágenes en distintos momentos del mismo.


\begin{figure*}[!ht]
\begin{center}
\includegraphics[width=13cm]{figures/otherbread}
\caption[Cortes de pan luego de la cocción, utilizando un campo escalar de dimensiones $256^{3}$]{Cortes de pan luego de la cocción, utilizando un campo escalar de dimensiones $256^{3}$. Las imágenes muestran que la miga y la corteza son consideradas en nuestro modelo.}
\label{fg:renders}
\end{center}
\end{figure*}

\begin{figure*}[!ht]
\begin{center}
\includegraphics[width=13cm]{figures/otherbread512}
\caption[Cortes de pan luego de la cocción, utilizando un campo escalar de dimensiones $512^{3}$]{Cortes de pan luego de la cocción, utilizando un campo escalar de dimensiones $512^{3}$. Se observan mayores detalles en la miga, lo cual aumenta el realismo de la imagen.}
\label{fg:renders2}
\end{center}
\end{figure*}

\begin{figure*}[!ht]
\begin{center}
\includegraphics[width=13cm]{figures/final}
\caption{Cortes de pan luego de la cocción con distintas geometrías.}
\label{fg:renders3}
\end{center}
\end{figure*}

\begin{figure*}[!ht]
\begin{center}
\includegraphics[width=13cm]{figures/croissant}
\caption[Cortes de una Croissant luego de la cocción]{Cortes de una Croissant luego de la cocción. La imagen muestra el interior de la miga en distintas regiones del volumen.}
\label{fg:croissant}
\end{center}
\end{figure*}

\begin{figure*}[!ht]
\begin{center}
\includegraphics[width=13cm]{figures/baked}
\caption[Cortes de pan luego de la cocción, con burbujas de mayor tamaño]{Cortes de pan luego de la cocción, con burbujas de mayor tamaño, usual en panes reales.}
\label{fg:bigalveoli}
\end{center}
\end{figure*}

\begin{figure*}[!ht]
\begin{center}
\includegraphics[width=13cm]{figures/bakedbunny}
\caption[Efecto de la cocción sobre el pan sintético.]{Cortes de pan luego de la cocción con burbujas de mayor tamaño. Además, puede observarse más claramente el efecto de la cocción en la deformación de las burbujas, las cuales tienden a seguir la forma de la corteza.}
\label{fg:bakedbunny}
\end{center}
\end{figure*}

\section{Conclusiones}
En este Cap\'itulo hemos introducido diversos algoritmos para subsanar la falta de procedimientos flexibles e intuitivos para modelar determinados materiales, entre ellos materiales porosos como el pan. Además, como sub-producto de la investigación, se obtuvieron representaciones bidimensionales de madera, granito, mármol y otros materiales comunes en la literatura.
Hemos demostrado visualmente la capacidad de los métodos.

El primer aporte constituye un algoritmo que produce texturas bidimensionales de materiales conocidos como madera y granito, utilizando sistemas de partículas, el cual fue presentado en una conferencia local \cite{Baravalle2011}.

Haciendo hincapié en materiales porosos, particularmente el pan, diseñamos algoritmos que utilizan sistemas de partículas en conjunción con Sistemas Dinámicos para producir texturas que representan de manera realista geometrías de pan, el cual fue presentado en un congreso y revista nacionales \cite{Baravalle2014}.

Finalmente, utilizando como referencia el proceso de formación real del pan, utilizamos conocimientos de ingeniería de los alimentos para producir una secuencia de pasos que emulan el proceso real de formación del pan.
El mismo permite modelar de manera realista tipos de panes arbitrarios, produciendo geometrías de su miga y su corteza, utilizando un proceso fractal de generación.
Para ello, emulamos un proceso de leudado configurable por el usuario, tanto en forma global, supliendo una geometría arbitraria en tres dimensiones, como en el burbujeado interno, por medio de parámetros intuitivos de generación (radio máximo y mínimo de las burbujas, cantidad de burbujas, etc.).
Luego aplicamos una simulación física de la cocción, la cual se utilizó para deformar ligeramente las burbujas, siguiendo las líneas definidas por la forma exterior del pan.
Finalmente, la corteza fue generada sobre la superficie del material.
El algoritmo de renderizado utilizado será explicado en el siguiente capítulo, el mismo produjo imágenes realistas de diversos tipos de pan, en varias etapas del proceso de formación del mismo.

El modelo introducido resulta más flexible y poderoso que aquéllos del estado del arte actual en modelado de geometría de materiales porosos y panes.

En un capítulo posterior, y con intención de validar el procedimiento, mostraremos que las geometrías obtenidas por el método inspirado en el proceso de formación del pan, arrojan distribuciones de burbujas coincidentes con panes reales.

